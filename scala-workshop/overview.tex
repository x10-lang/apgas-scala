\section{Overview of \apgas in Scala}
\label{sec:apgas}

\paragraph{Terminology.}
A {\em place} is an abstraction of a mutable, shared-memory region and worker threads operating on this memory.
A single application typically runs over a collection of places. In this work, each place is implemented as a separate JVM.

A {\em task} is an abstraction of a sequence of computations. In this work, a task is specified as a block.
Each task is bound to a particular place. 
A task can spawn local and remote tasks, i.e., tasks to be executed in the same place or elsewhere.

A local task shares the heap of the parent task. A remote task executes on a snapshot of the parent task's heap captured when the task is spawned. A task can instantiate \emph{global references} to objects in its heap to work around the capture semantics.
Global references are copied as part of the snapshot but not the target objects. A global reference can only be dereferenced
at the place of the target object where it resolves to the original object.
% FIXME how does one describe GlobalRef.forPlaces in these words?

A task can wait for the termination of all the tasks transitively spawned from it.
Thanks to global references, remote tasks, and termination control,
a task can indirectly manipulate remote objects.

\paragraph{Constructs.}
The two fundamental control structures in APGAS are
 \lstinline{asyncAt}, and \lstinline{finish}, whose signatures in
the Scala implementation are:
\begin{lstlisting}
  def asyncAt(place: Place)(body: $\RA$Unit) : Unit
  def finish(body: $\RA$Unit) : Unit
\end{lstlisting}
As is common in Scala libraries, we use by-name arguments to capture blocks.

The \lstinline{asyncAt} construct spawns an asynchronous task at place \lstinline{p} and returns
immediately. It is therefore the primitive construct for both \emph{concurrency} and \emph{distribution}.
The \lstinline{finish} construct detects termination: an invocation of
\lstinline{finish} will execute its body and then block until all nested invocations
of \lstinline{asyncAt} have completed. The set of \lstinline{asyncAt} invocations
that are controlled includes all recursive invocations, including all remote
ones. This makes \lstinline{finish} a powerful contribution of \apgas.

Because spawning local tasks is so common, the library defines an optimized version of
\lstinline{asyncAt} for this purpose with the signature:
\begin{lstlisting}
  def async(body: $\RA$Unit) : Unit
\end{lstlisting}
We can use \lstinline{async} for local concurrency. For instance, a parallel
version of a Fibonacci number computation can be expressed as:
\begin{lstlisting}
  def fib(i: Int) : Long = if(i $\SLE$ 1) i else {
    var a, b: Long = 0L
    finish {
      async { a = fib(i - 2) }
      b = fib(i - 1)
    }
    a + b
  }
\end{lstlisting}
In the code above, each recursive invocation of \lstinline{fib} spawns an
additional asynchronous task, and \lstinline{finish} blocks until all
recursive dependencies have been computed.

Another common pattern is to execute a computation remotely and block until the
desired return value is available. For this purpose, the library defines:
\begin{lstlisting}
  def at[T:Serialization](place: Place)(body: $\RA$T) : T
\end{lstlisting}
(We discuss the \lstinline{Serialization} type class in
Section~\ref{sec:serialization}.)

\paragraph{Messages and place-local memory.}

Transferring data between places is achieved by capturing the relevant part of
the sender's heap in the body of the \lstinline{asyncAt} block. In many
situations, however, it is convenient to refer to a section of the memory that
is \emph{local} to a place using a \emph{global} name common to all places. For this
purpose, the library defines the \lstinline{PlaceLocal} trait. In an
application that defines one \lstinline{Worker} object per place, for instance,
we can write:
\begin{lstlisting}
  class Worker($\ldots$) extends PlaceLocal
\end{lstlisting}
Initializing an independent object at each place is achieved using the
\lstinline{forPlaces} helper:
\begin{lstlisting}
  val w = PlaceLocal.forPlaces(places) { new Worker() }
\end{lstlisting}
At this stage, the variable \lstinline{w} holds a proper instance of
\lstinline{Worker}. The important property of place-local objects is reflected
in the following code:
\begin{lstlisting}
  asyncAt(p2) {
    w.work($\ldots$)
  }
\end{lstlisting}
When serializing the instance of \lstinline{PlaceLocal} that belongs to the
closure, the runtime replaces the \lstinline{Worker} object by a named
reference. When the closure is deserialized at the destination place
\lstinline{p2}, the reference is resolved to the \emph{local} instance of
\lstinline{Worker} and the work is executed using the memory local to
\lstinline{p2}.

For a type \lstinline{T} that cannot extend \lstinline{PlaceLocal}, the library
defines \lstinline{GlobalRef[T]}, which acts as a wrapper.\footnote{The name
comes from the fact that a \lstinline{GlobalRef} is available globally,
even though it points to place-local objects.}
We use its method \lstinline{apply(): T} to access the
wrapped value local to each place.

\paragraph{Handling failures.} Remote invocations can fail, for instance if the
code throws an exception or if the process hosting the place terminates
unexpectedly. The error handling model of APGAS is to surface errors up to the
first enclosing \lstinline{finish}, which throws an exception. The critical
property that APGAS maintains is \emph{happens-before invariance}: failures
cannot introduce execution orderings that are not possible under regular
execution conditions \cite{ppopp14,ecoop14}. Detailed examples of resilient
benchmarks are beyond the scope of this short paper.

% \subsection{Similarities and Differences with Futures and Actors}
% 
% Discuss future vs.\ async (returning void means no value). Future can be awaited anywhere, but must be awaited ``manually''.
% 
% finish counts (including distributed tasks!) and has no real equivalent.
% 
% Actors: unified model for concurrency and distribution.
% 
% Handling failures: Try vs Exception vs. Messages
% 
% Async at: active messages: get executed "immediately", actor messages handled one-by-one.
% 

In the following sections, we highlight some \apgas patterns in two concrete
benchmarks.
