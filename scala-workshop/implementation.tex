\section{Implementation Status}
\label{sec:serialization}

The \apgas library is implemented in about 2,000 lines of Java 8 code, with a
Scala wrapper of about 200 lines. It uses the fork/join framework for
scheduling tasks in each place. The library exposes its
\lstinline{ExecutorService}, making it possible in principle to develop applications that
use \apgas in cooperation with Scala futures.
Distribution is built on top of the Hazelcast in-memory
data grid~\cite{hazelcast}. \apgas relies on Hazelcast to
assemble clusters of JVMs and invoke remote tasks. 

The Scala layer defines the \lstinline{Serialization} type class as a mechanism
to handle all Scala types uniformly, converting them to types
compatible with \lstinline{java.io.Serializable}, as required by Hazelcast. An
alternative would be to bypass Java serialization entirely and use, e.g.,
pickling
\cite{MillerETAL13InstantPicklesGeneratingObjectorientedPicklerCombinators}.

Another possible improvement is the handling of capture in closures:
environment capture is a mechanism central to \apgas, but is error prone. The
problem is well-known and the X10 compiler, for instance, handles it with
custom warnings. In \apgas for Scala, using spores with properly defined
headers
\cite{MillerHallerOdersky14SporesTypebasedFoundationClosuresAgeConcurrency}
would help clarify the movement of data between places.

% APGAS supports elasticity. Places can be added to a running application by simply launching a new JVM with the \emph{ip:port} address of an existing JVM in the cluster. 
% Applications can register a callback that is invoked when a place is added or has failed.

%The APGAS library implements the core elements of the APGAS programming model: lightweight tasks, distributed termination detection, and global heap references. Exceptions escaping from tasks are collected by the innermost enclosing finish.
%By setting the \emph{apgas.resilient} system property at startup time, the application can request resilient versions of these core elements. Remote task invocations fail gracefully when the destination place is unavailable. Resilient finish ensures \emph{happen-before invariance}~\cite{ppopp14,ecoop14}.

%For convenience, we implement two alternative launchers to start multiple places at once either on the localhost or, using Hadoop YARN, in a distributed system.
%

%The APGAS library is currently implemented in about 2,000 non-blank, non-comment lines of Java code. About a third of this code implements the resilient and non-resilient distributed termination detection algorithms. In comparison the X10 compiler, runtime, and standard library code bases comprise more than 200,000 lines of X10, Java, and C++ code.\footnote{Data generated using David A. Wheeler's SLOCCount.}

% \paragraph{Serialization.}
% Use of type class, mostly to support primitive types and
% \lstinline{java.io.Serializable} together. In principle, paves the way for
% relying on pickling \cite{MillerETAL13InstantPicklesGeneratingObjectorientedPicklerCombinators} in future work.
% 
% \paragraph{Closures and capture.} Currently not checking. Ideal use case for spores \cite{MillerHallerOdersky14SporesTypebasedFoundationClosuresAgeConcurrency}.
