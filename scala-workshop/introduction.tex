\section{Introduction}

\apgas is an extension of PGAS. Implemented as the foundation of the X10
programming language. X10 has a JVM and native backend. X10, and by extension
\apgas, has been used successfully to run parallel and distributed applications
on tens of thousands of cores \cite{TardieuETAL14X10ApgasAtPetascale}.

Current state requires programmers interested in the \apgas model to buy in the
X10 language and development platform, which may not be possible for an array
of reasons.

To expose more programmers, we believe in the approach of exposing the critical
components of \apgas as a library. Particularly suitable for Scala, a language
that is welcoming of library-based extensions, and that has been a pioneer on
the JVM for alternative concurrency paradigms. Notably, the original actor
library and its more recent successor Akka are examples of both. Our work is
based on a recent implementation of \apgas in Java \cite{APGASJava}, reworked to
fit Scala idioms.

We hope that people will embrace \apgas, either as a substitute for or as a
complement to other concurrency libraries. Examples throughout the paper are
verbatim Scala/\apgas code. X10 shares ancestry/inspiration with Scala, and the
facilities in Scala for library-defined language extensions makes the code look
almost exactly like X10 programs.

\subsection{Overview of the \apgas Programming Model}

Terminology: places, tasks.

Memory model.

The three fundamental control structures in programming model are
\lstinline{async}, \lstinline{at}, and \lstinline{finish}, whose signatures in
the Scala implementation are:
\begin{lstlisting}
  def async(body: $\RA$Unit) : Unit
  def at[T:Serialization](place: Place)(body: $\RA$T) : T
  def finish(body : $\RA$Unit) : Unit
\end{lstlisting}
As is common in Scala libraries, we use by-name arguments to capture blocks.
We discuss the \lstinline{Serialization} type class in
Section~\ref{sec:serialization}.

The \lstinline{async} construct spawns an asynchronous task and returns
immediately. It is therefore the primitive construct for \emph{concurrency}.

The primitive construct for \emph{distribution} is \lstinline{at}. The invocation
\begin{lstlisting}
  val r = at(p) { work() }
\end{lstlisting}
executes the method \lstinline{work} at place \lstinline{p}, and blocks until
it terminates and returns the result. Asynchronously spawning tasks at remote places can be achieved by composing the two constructs:
\begin{lstlisting}
  at(p) { async { work() }}
\end{lstlisting}
This operation is so common that we have an semantically equivalent but optimized version with the signature
\begin{lstlisting}
  def asyncAt(place: Place)(body: $\RA$Unit) : Unit
\end{lstlisting}

Finally, \lstinline{finish} detects termination: an invocation of
\lstinline{finish} will execute its body and block until all nested invocations
of \lstinline{async} have completed. The set of \lstinline{async} invocations
that are controlled includes all recursive invocations, including all remote
ones. This makes \lstinline{finish} a powerful contribution of \apgas.

Fibonacci

SPDM

Memory: PlaceLocal, PlaceLocalRef

Place failure.

\subsection{Similarities and Differences with Futures and Actors}

Discuss future vs.\ async (returning void means no value). Future can be awaited anywhere, but must be awaited ``manually''.

finish counts (including distributed tasks!) and has no real equivalent.

Actors: unified model for concurrency and distribution.

Handling failures: Try vs Exception vs. Messages

Async at: active messages: get executed "immediately", actor messages handled one-by-one.

In the following sections, we highlight some of the points above through two
benchmarks, \kmeans clustering and unbalanced tree search.


% \newsavebox\patternRemoteEval
\begin{lrbox}{\patternRemoteEval}
\begin{lstlisting}
v = at(p) {
  evalThere(arg1, arg2)
}
\end{lstlisting}
\end{lrbox}

\newsavebox\patternActiveMessage
\begin{lrbox}{\patternActiveMessage}
\begin{lstlisting}
asyncAt(p) {
  runThere(arg1, arg2)
}
\end{lstlisting}
\end{lrbox}

\newsavebox\patternRecursiveParallelDecomposition
\begin{lrbox}{\patternRecursiveParallelDecomposition}
\begin{lstlisting}
def fib(n : Long) : Long = {
  if(n < 2) n else {
    var f1, f2 = 0L
    finish {
      async { f1 = fib(n - 1) }
      f2 = fib(n - 2)
    }
    f1 + f2
  }
}
\end{lstlisting}
\end{lrbox}

\newsavebox\patternSPMD
\begin{lrbox}{\patternSPMD}
\begin{lstlisting}
finish {
  for(p $\lA$ places) {
    at(p) {
      async { runEverywhere(arg1, arg2) }
    }
  }
}
\end{lstlisting}
\end{lrbox}

\begin{figure*}
\centering
\subfloat{\usebox\patternRemoteEval}\qquad
\subfloat{\usebox\patternActiveMessage}\qquad
\subfloat{\usebox\patternRecursiveParallelDecomposition}\qquad
\subfloat{\usebox\patternSPMD}\qquad
\caption{Common \apgas patterns in Scala.\label{fig:apgas-patterns}}
\end{figure*}

% Figure~\ref{fig:apgas-patterns} shows patterns.
