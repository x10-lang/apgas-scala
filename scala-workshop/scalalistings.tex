%% To import in the preambule
%\usepackage{listings}

% "define" Scala
\lstdefinelanguage{scala}{
  alsoletter={@,=,>},
  morekeywords={abstract, case, catch, class, def, do, else, extends, false, final, finally, for, foreach, if, implicit, import, match, new, null, object, 
override, package, private, protected, requires, return, sealed, super, this, throw, trait, try, true, type, val, var, while, with, yield, domain, 
postcondition, precondition,invariant, constraint, assert, forAll, in, _, async, asyncAt, finish, at},
  sensitive=true,
  morecomment=[l]{//},
  morecomment=[s]{/*}{*/},
  morestring=[b]"
}

\newcommand{\codestyle}{\small\sffamily}

\newcommand{\SAND}{\mbox{\tt \&\&}\xspace}
\newcommand{\SOR}{\mbox{\tt ||}\xspace}
\newcommand{\MOD}{\mbox{\tt \%}\xspace}
\newcommand{\DIV}{\mbox{\tt /}\xspace}
\newcommand{\PP}{\mbox{\tt ++}\xspace}
\newcommand{\PL}{\mbox{\tt +}\xspace}
\newcommand{\MM}{\mbox{\tt {-}{-}}\xspace}
\newcommand{\RA}{\Rightarrow}
\newcommand{\EQ}{\mbox{\tt ==}}
\newcommand{\NEQ}{\mbox{\tt !=}}
\newcommand{\SLE}{\ensuremath{\leq}}
\newcommand{\SGE}{\ensuremath{\geq}}
\newcommand{\SGT}{\mbox{\tt >}}
\newcommand{\SLT}{\mbox{\tt <}}
\newcommand{\rA}{\rightarrow}
\newcommand{\lA}{\leftarrow}

% Default settings for code listings
\lstset{
  language=scala,
  showstringspaces=false,
  columns=fullflexible,
  mathescape=true,
  numbers=none,
  numberstyle=\tiny,
  basicstyle=\codestyle
} 

