\section{Distributed \kmeans Clustering}
\label{sec:kmeans}

The \kmeans benchmark uses Lloyd's algorithm~\cite{Lloyd1982Least} to divide a
set of points in a $d$-dimensional space into $k$ disjoint clusters.  Given an
arbitrary set of initial clusters, the algorithm iterates over the following
steps:
\begin{enumerate}
  \item For each point, assign that point to whichever cluster is closest (by
Euclidean distance to the cluster centroid).
  \item For each cluster, update the centroid (the arithmetic mean of all
points assigned to that cluster).
\end{enumerate}
Distributed computation is straightforward: each process holds a portion of the
points and computes cluster assignments and centroid contributions for each
point. At each iteration, a master process collects all centroid contributions,
computes the aggregates, and communicates the updated values to all workers.

% \begin{lstlisting}
%   for(iter $\lA$ 0 until maxIter if !converged()) {
%     finish {
%       for (p $\lA$ places) {
%         async {
%           val placeResult = at(p) { matchPoints() }
%           synchronized {
%             for (i $\lA$ 0 until C; j $\lA$ 0 until D) {
%               newCentroids(i)(j) += placeResult.centroids(i)(j)
%             }
%             for (j $\lA$ 0 until C) {
%               newCounts(j) += placeResult.counts(j)
%             }
%           }
%     }}}
% 
%     newCentroids /= newCounts
%     centroids = newCentroids
%     counts = 0
%   }
% \end{lstlisting}
 
Figure~\ref{fig:kmeansCode} shows the code for a distributed \kmeans computation with the APGAS library.
Each process holds a portion of the points and computes cluster assignments and centroid contributions for each point; in addition, place 0 maintains the current set of cluster centroids.
In each iteration, an \lstinline{async} task is created for each place.
The \lstinline{at} construct is used to compute cluster contributions remotely at the place; \lstinline{at} automatically captures the current set of \lstinline{centralClusters} and sends them as part of the message to the remote place.
The place contributions to the central clusters are returned to place 0 on completion of the remote computation.
The \lstinline{finish} construct ensures that contributions for all places are computed before continuing with the next iteration.

\begin{figure}[tbp]
  \begin{lstlisting}
class ClusterState extends Serializable {
  val clusters = Array.ofDim[Float](K, D)
  val clusterCounts = Array.ofDim[Int](K)
}
val clusterState = PlaceLocalRef.forPlaces(places) {
  new ClusterState()
}
val centralClusters = Array.ofDim[Float](K, D)
val centralClusterCounts = Array.ofDim[Int](K)

while (!converged()) {
  finish {
    for (place <- places) {
      async {
        val placeContributions = at(place) {
          val clusters = clusterState().clusters
          val counts = clusterState().counts

          for (p <- 0 until ps.length) {
            val c = closestCentroid(points(p), centralClusters)
            for (d <- 0 until D) {
              clusters(c)(d) += points(p)(d)
            }
            counts(c) += 1
          }
          clusterState
        }
        // Add place contributions to centralClusters
        // Add place contributions to centralClusterCounts
      }
    }
  }
}  

  \end{lstlisting}
  \caption{APGAS code for \kmeans}
  \label{fig:kmeansCode}
\end{figure}

% Problem definition. Why it's ``easy'' to distribute.
% 
% Contrast with actors: messaging and communication is implicit vs. explicit.
% 
% Shared memory is not idiomatic in Akka, whereas embraced in APGAS. Ex: updating
% the central array. This means we end up with code that requires concurrency
% control.
% 
% Here we just used \lstinline{synchronized}. If Java had a proper
% implementation or arrays with atomic increments for floats (analogous to
% \lstinline{AtomicIntegerArray}), we could also just use that.
